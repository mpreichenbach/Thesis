\begin{abstract}
		Ecologists have recently used integral projection models (IPMs) to study fish and other animals which continue to grow throughout their lives. Such animals cannot shrink, since they have bony skeletons; a mathematical consequence of this is that the kernel of the integral projection operator $T$ is unbounded, and the operator is not compact. \emph{A priori}, it is unclear whether these IPMs have an asymptotic growth rate $\lambda$, or a stable-stage distribution $\psi$. In the case of a compact operator, these quantities are its spectral radius and the associated eigenvector, respectively. Under biologically reasonable assumptions, we prove that the non-compact operators in these IPMs share important spectral properties with their compact counterparts. Specifically, we show that the operator $T$ has a unique positive eigenvector $\psi$ corresponding to its spectral radius $\lambda$, the spectral radius $\lambda$ is strictly greater than the supremum of all other spectral values, and for any nonnegative initial population $\varphi_0$, there is a $c>0$ such that $T^n\varphi_0/\lambda^n \to c \cdot \psi$. We also show that the zeros of certain functions defined by sums of compact operators can be used to approximate the spectral radius $\lambda$ of the non-compact operator $T$. In the final chapter, we give some simulations showing the long-term behavior of a density-dependent IPM.
\end{abstract}
